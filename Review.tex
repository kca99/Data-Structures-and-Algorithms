\documentclass{article}

\usepackage{listings}
\usepackage{xcolor}

\title{Data Structures \& Algorithms - A Review}
\author{Kevin Chung}
\date{December 2016}

\lstset { %
	language=C++,
	backgroundcolor=\color{black!5}, % set backgroundcolor
	basicstyle=\footnotesize,% basic font setting
	xleftmargin=.25in,
	xrightmargin=.25in,
	numbersep=5pt
}
\begin{document}

\begin{titlepage}
\maketitle
\thispagestyle{empty}
\end{titlepage}

\newpage
\tableofcontents
\thispagestyle{empty}
\newpage

\clearpage
\setcounter{page}{1}

\section{Preface}
This review package/compilation/set or whatever you wish to call it is a way for me, Kevin to review for my coming semesters of computing science courses. These are the fundamentals of which I have learned in my past couple years of undergraduate education. There may be errors, I'm not perfect but I will try my best to convey the most accurate information. Most of the information will be comprised from my personal notes or from a professor's notes. If I find that information is lacking on a topic or I don't understand it enough, the Internet may prove to be useful. 

\section{Running Time}
One of the earliest concepts programmers learn is running time specifically in regards to algorithms and comparing algorithms. Although there are many types of notations to describe run time, I will be using Big O notation as this describes for lack of a better time, the worst case scenario or the upper bound on the growth rate of a function. For a very primitive example, lets iterate through a whole array of numbers and print them out. 

\subsection{Example: Printing an Array}
\begin{lstlisting}
int prntArr(int Arr[], int n){
	for(int i=0; i<n; i++){
		std::cout<<Arr[i]<<std::endl;
	}
}
\end{lstlisting}

\noindent The above code simply iterates through the array via the for loop using the indices from 0 to n (the size of the array). As a result, if we had an array of size n, and arbitrary number this time, we would have to print out n numbers. Thus, the upper bound of this algorithm would be O(n) or Big-O of n as there can be no greater number of outputs larger than n. In another way of analyzing it, we only call the cout function n times at most. The O in Big-O stands for Order; we will see another example of it in the next example. 
\\
\subsection{Example: Bubble Sort}
Personally, when I started my first Computing science course, I wasn't thinking about optimization. I was focused on getting the right answer and always amazed at the results. I remember very clearly how I would use bubble sort (I find this pretty funny after years of programming) to sort a list of numbers. In this example, I will analyze bubblesort(). 

\newpage
\begin{lstlisting}
void bubblesort(int list[], int n){
	for(int i=0; i<n-1; i++){
		for(int j=0; j<n-1-i; j++){
			if(list[j]>list[j+1]){
				int tempval = list[j];
				list[j] = list[j+1];
				list[j+1] = tempval;
			}
		}
	}
}
\end{lstlisting}

\noindent In the above code, we have parameters list[] and an int n which are respectively the list we wish to sort and the size n. 
\section{Data Structures}

\subsection{Lists}
\subsection{Stacks}
\subsection{Queues}
\
\end{document}